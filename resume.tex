%% start of file `template.tex'.
%% Copyright 2006-2013 Xavier Danaux (xdanaux@gmail.com).
%
% This work may be distributed and/or modified under the
% conditions of the LaTeX Project Public License version 1.3c,
% available at http://www.latex-project.org/lppl/.


\documentclass[10pt,letterpaper,roman]{moderncv}        % possible options include font size ('10pt', '11pt' and '12pt'), paper size ('a4paper', 'letterpaper', 'a5paper', 'legalpaper', 'executivepaper' and 'landscape') and font family ('sans' and 'roman')

% modern themes
\moderncvstyle{banking}                            % style options are 'casual' (default), 'classic', 'oldstyle' and 'banking'
\moderncvcolor{blue}                                % color options 'blue' (default), 'orange', 'green', 'red', 'purple', 'grey' and 'black'
%\renewcommand{\familydefault}{\sfdefault}         % to set the default font; use '\sfdefault' for the default sans serif font, '\rmdefault' for the default roman one, or any tex font name
\nopagenumbers{}                                  % uncomment to suppress automatic page numbering for CVs longer than one page

% character encoding
\usepackage[utf8]{inputenc}
\usepackage{fontawesome}
\usepackage{fontspec}
\usepackage{tabularx}
\usepackage{ragged2e}
% if you are not using xelatex ou lualatex, replace by the encoding you are using
%\usepackage{CJKutf8}                              % if you need to use CJK to typeset your resume in Chinese, Japanese or Korean

% adjust the page margins
\usepackage[scale=0.8]{geometry}
\usepackage{multicol}
%\setlength{\hintscolumnwidth}{3cm}                % if you want to change the width of the column with the dates
%\setlength{\makecvtitlenamewidth}{10cm}           % for the 'classic' style, if you want to force the width allocated to your name and avoid line breaks. be careful though, the length is normally calculated to avoid any overlap with your personal info; use this at your own typographical risks...

\usepackage{import}

% personal data
\name{Aleksei}{Sholokhov}
% \title{Curriculum Vitae}                               % optional, remove / comment the line if not wanted
\address{202 Lewis Hall, Seattle, WA 98105}{}{}% optional, remove / comment the line if not wanted; the "postcode city" and and "country" arguments can be omitted or provided empty
% \phone[mobile]{909-839-3097}                   % optional, remove / comment the line if not wanted
% \phone[fixed]{01234 123456}                    % optional, remove / comment the line if not wanted
%\phone[fax]{+3~(456)~789~012}                      % optional, remove / comment the line if not wanted
% \email{xpan1@swarthmore.edu}                               % optional, remove / comment the line if not wanted
% \homepage{shawnpan.me}                         % optional, remove / comment the line if not wanted
% \extrainfo{}                 % optional, remove / comment the line if not wanted
%\photo[64pt][0.4pt]{picture}                       % optional, remove / comment the line if not wanted; '64pt' is the height the picture must be resized to, 0.4pt is the thickness of the frame around it (put it to 0pt for no frame) and 'picture' is the name of the picture file
%\quote{Some quote}                                 % optional, remove / comment the line if not wanted

% to show numerical labels in the bibliography (default is to show no labels); only useful if you make citations in your resume
%\makeatletter
%\renewcommand*{\bibliographyitemlabel}{\@biblabel{\arabic{enumiv}}}
%\makeatother
%\renewcommand*{\bibliographyitemlabel}{[\arabic{enumiv}]}% CONSIDER REPLACING THE ABOVE BY THIS

% bibliography with mutiple entries
%\usepackage{multibib}
%\newcites{book,misc}{{Books},{Others}}
  
\newcommand*{\customcventry}[7][.25em]{
  \begin{tabular}{@{}l} 
    {\bfseries #4}
  \end{tabular}
  \hfill% move it to the right
  \begin{tabular}{l@{}}
     {\bfseries #5}
  \end{tabular} \\
  \begin{tabular}{@{}l} 
    {\itshape #3}
  \end{tabular}
  \hfill% move it to the right
  \begin{tabular}{l@{}}
     {\itshape #2}
  \end{tabular}
  \ifx&#7&%
  \else{\\%
    \begin{minipage}{\maincolumnwidth}%
      \small#7%
    \end{minipage}}\fi%
  \par\addvspace{#1}}

\newcommand*{\customcvproject}[4][.25em]{
%   \vfill\noindent
  \begin{tabular}{@{}l} 
    {\bfseries #2}
  \end{tabular}
  \hfill% move it to the right
  \begin{tabular}{l@{}}
     {\itshape #3}
  \end{tabular}
  \ifx&#4&%
  \else{\\%
    \begin{minipage}{\maincolumnwidth}%
      \small#4%
    \end{minipage}}\fi%
  \par\addvspace{#1}}

\setlength{\tabcolsep}{12pt}

%----------------------------------------------------------------------------------
%            content
%----------------------------------------------------------------------------------
\begin{document}
%\begin{CJK*}{UTF8}{gbsn}                          % to typeset your resume in Chinese using CJK
%-----       resume       ---------------------------------------------------------
\makecvtitle
\vspace*{-23mm}

\begin{center}
\begin{tabular}{ c c c c }
 \faGlobe\enspace linkedin.com/in/aksholokhov & \faEnvelopeO\enspace aksh@uw.edu & \faGithub\enspace aksholokhov & \faMobile\enspace 505-557-59-81\\  
\end{tabular}
\end{center}

\section{EDUCATION}
{\customcventry{Expected Graduation: 06/2023}{Ph.D. in Applied Mathematics}{University of Washington}{Seattle, WA}{}{}}

\section{SKILLS}

\customcventry{09/2018-now}{As Research Assistant at the University of Washington}{Research in Machine Learning Algorithms, Deep Learning, and Optimization}{Seattle, WA}{}
{\begin{itemize}
	\item Integrated \texttt{Neural ODE} techniques into model discovery library \texttt{SINDy}, under the supervision of J. Nathan Kutz and Steve Brunton. Decreased the amount of data needed to model noisy dynamical systems by a factor of 10.
	\item Accelerated training of deep neural networks using higher-order optimization methods. Implemented it as \texttt{TensorFlow} and \texttt{jax} modules. Achieved state-of-the-art performance for selected image recognition tasks.
\end{itemize}
}

\customcventry{09/2019-12/2021}{As Research Assistant at the Institute for Health Metrics and Evaluation (IHME)}{Data Science and Statistical Analysis}{Seattle, WA}{}
{\begin{itemize}
	\item Invented new statistical modeling tool \texttt{pysr3} which does feature selection using non-convex optimization techniques. Implemented it as a  \texttt{scikit-learn}-compatible \texttt{python} package. Achieved 30-fold speed-up relative to the competitors.
	\item Developed a statistical model that projects cases and deaths from COVID-19 in collaboration with a team of 130 researchers. It helped governmental decision makers manage resources and plan ahead during the pandemic.
\end{itemize}
}


\customcventry{09/2018-now}{As Research Assistant at University of Washington and IHME}{Software Development in Python, MATLAB, and C++}{Seattle, WA}{}
{\begin{itemize}
    \item Developed \texttt{gspack}: python-autograder to accelerate grading of coding assignments. This package is successfully used for 5 scientific computing classes for thousands of homeworks in Department of Applied Mathematics.
 	\item Enabled SVM classifiers to work with large-scale data using approximate nearest neighbor search. Implemented it using \texttt{SQL}, \texttt{C++}, and \texttt{Python}. Improved accuracy and memory costs by 30\%  over competitors.
 	\item Learned \texttt{OpenMP}, \texttt{MPI}, \texttt{CUDA}, and \texttt{MATLAB} by working as a teaching assistant for graduate-level High-Performance Computing and Scientific Computing classes for 6 quarters.  
\end{itemize}
}

\customcventry{02/2016-07/2018}{As Research Student at Computing Center of Russian Academy of Science}{Project Management, Communication, and Leadership skills}{Moscow, Russia}{}
{\begin{itemize}
    \item Led \texttt{RySearch} project: an exploratory data analysis and recommender system that simplifies knowledge discovery with NLP techniques such as topic modeling. Implemented it using \texttt{python}, \texttt{JavaScript}, and \texttt{MongoDB}.
    \item Effectively organized research and software development in a team of 4 researchers. Published 2 novel quality metrics for topic models based on this work.
   	\end{itemize}
} 

\customcventry{09/2020 - Now}{As a Diversity, Equity, and Inclusion (DEI) Committee Member at UW}{Negotiation Skills, Cross-Functional Collaboration, and Cross-Cultural Dialog}{Seattle, WA}{}
{\begin{itemize}
	\item Developed 10-year Diversity Action Plan for the Department of Applied Mathematics.
	\item Negotiated \$20k financial commitment from the department of Applied Mathematics to Early Scholars Program.   
	\item Organized and led climate orientations and educational seminars on importance of diversity and inclusion in academia.
\end{itemize}
}


\section{SELECTED PUBLICATIONS}
 
 \begin{itemize}
 		\item Sholokhov A., Santomauro D., Burke J., Zheng P., and Aravkin A., "Universal Feature Selection for Mixed-Effects Models with Non-Convex Penalties", \textit{in preparation}
 		\item Sholokhov, A., Zheng, P., and Aravkin, A., "pysr3: Python Library for Sparse Relaxed Regularized Regression", \textit{under peer-review}
        \item IHME Covid-19 Forecasting Team, "Modeling COVID-19 scenarios for the United States". \textit{Nature Medicine, 2020}
        \item Belyy A.V., Selezniova, M.S., Sholokhov, A., and Vorontsov, K., "Quality Evaluation and Improvement for Hierarchical Topic Modelling",  \textit{24rd International Conference on Computational Linguistics}
    \end{itemize}


% Publications from a BibTeX file without multibib
%  for numerical labels: \renewcommand{\bibliographyitemlabel}{\@biblabel{\arabic{enumiv}}}% CONSIDER MERGING WITH PREAMBLE PART
%  to redefine the heading string ("Publications"): \renewcommand{\refname}{Articles}
%\nocite{*}
%\bibliographystyle{plain}
%\bibliography{publications}                        % 'publications' is the name of a BibTeX file

% Publications from a BibTeX file using the multibib package
%\section{Publications}
%\nocitebook{book1,book2}
%\bibliographystylebook{plain}
%\bibliographybook{publications}                   % 'publications' is the name of a BibTeX file
%\nocitemisc{misc1,misc2,misc3}
%\bibliographystylemisc{plain}
%\bibliographymisc{publications}                   % 'publications' is the name of a BibTeX file

%-----       letter       ---------------------------------------------------------

\end{document}


%% end of file `template.tex'.
