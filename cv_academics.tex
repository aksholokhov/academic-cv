%%%%%%%%%%%%%%%%%%%%%%%%%%%%%%%%%%%%%%%%%%%%%%%%%%%%%%%%%%%%%%%%%%%%%%%%
%%%%%%%%%%%%%%%%%%%%%% Simple LaTeX CV Template %%%%%%%%%%%%%%%%%%%%%%%%
%%%%%%%%%%%%%%%%%%%%%%%%%%%%%%%%%%%%%%%%%%%%%%%%%%%%%%%%%%%%%%%%%%%%%%%%

%%%%%%%%%%%%%%%%%%%%%%%%%%%%%%%%%%%%%%%%%%%%%%%%%%%%%%%%%%%%%%%%%%%%%%%%
%% NOTE: If you find that it says                                     %%
%%                                                                    %%
%%                           1 of ??                                  %%
%%                                                                    %%
%% at the bottom of your first page, this means that the AUX file     %%
%% was not available when you ran LaTeX on this source. Simply RERUN  %%
%% LaTeX to get the ``??'' replaced with the number of the last page  %%
%% of the document. The AUX file will be generated on the first run   %%
%% of LaTeX and used on the second run to fill in all of the          %%
%% references.                                                        %%
%%%%%%%%%%%%%%%%%%%%%%%%%%%%%%%%%%%%%%%%%%%%%%%%%%%%%%%%%%%%%%%%%%%%%%%%

%%%%%%%%%%%%%%%%%%%%%%%%%%%% Document Setup %%%%%%%%%%%%%%%%%%%%%%%%%%%%

% Don't like 10pt? Try 11pt or 12pt
\documentclass[10pt]{article}

% The automated optical recognition software used to digitize resume
% information works best with fonts that do not have serifs. This
% command uses a sans serif font throughout. Uncomment both lines (or at
% least the second) to restore a Roman font (i.e., a font with serifs).
%\usepackage{times}
%\renewcommand{\familydefault}{\sfdefault}

% This is a helpful package that puts math inside length specifications
\usepackage{calc}
\usepackage{comment}

% Simpler bibsection for CV sections
% (thanks to natbib for inspiration)
\makeatletter
\newlength{\bibhang}
\setlength{\bibhang}{1em} %1em}
\newlength{\bibsep}
 {\@listi \global\bibsep\itemsep \global\advance\bibsep by\parsep}
\newenvironment{bibsection}%
        {\begin{enumerate}{}{%
%        {\begin{list}{}{%
       \setlength{\leftmargin}{\bibhang}%
       \setlength{\itemindent}{-\leftmargin}%
       \setlength{\itemsep}{\bibsep}%
       \setlength{\parsep}{\z@}%
        \setlength{\partopsep}{0pt}%
        \setlength{\topsep}{0pt}}}
        {\end{enumerate}\vspace{-.6\baselineskip}}
%        {\end{list}\vspace{-.6\baselineskip}}
\makeatother

% Layout: Puts the section titles on left side of page
\reversemarginpar

%
%         PAPER SIZE, PAGE NUMBER, AND DOCUMENT LAYOUT NOTES:
%
% The next \usepackage line changes the layout for CV style section
% headings as marginal notes. It also sets up the paper size as either
% letter or A4. By default, letter was used. If A4 paper is desired,
% comment out the letterpaper lines and uncomment the a4paper lines.
%
% As you can see, the margin widths and section title widths can be
% easily adjusted.
%
% ALSO: Notice that the includefoot option can be commented OUT in order
% to put the PAGE NUMBER *IN* the bottom margin. This will make the
% effective text area larger.
%
% IF YOU WISH TO REMOVE THE ``of LASTPAGE'' next to each page number,
% see the note about the +LP and -LP lines below. Comment out the +LP
% and uncomment the -LP.
%
% IF YOU WISH TO REMOVE PAGE NUMBERS, be sure that the includefoot line
% is uncommented and ALSO uncomment the \pagestyle{empty} a few lines
% below.
%

%% Use these lines for letter-sized paper
\usepackage[paper=letterpaper,
            %includefoot, % Uncomment to put page number above margin
            marginparwidth=1.2in,     % Length of section titles
            marginparsep=.05in,       % Space between titles and text
            margin=1in,               % 1 inch margins
            includemp]{geometry}

%% Use these lines for A4-sized paper
%\usepackage[paper=a4paper,
%            %includefoot, % Uncomment to put page number above margin
%            marginparwidth=30.5mm,    % Length of section titles
%            marginparsep=1.5mm,       % Space between titles and text
%            margin=25mm,              % 25mm margins
%            includemp]{geometry}

%% More layout: Get rid of indenting throughout entire document
\setlength{\parindent}{0in}

\usepackage[shortlabels]{enumitem}

%% Reference the last page in the page number
%
% NOTE: comment the +LP line and uncomment the -LP line to have page
%       numbers without the ``of ##'' last page reference)
%
% NOTE: uncomment the \pagestyle{empty} line to get rid of all page
%       numbers (make sure includefoot is commented out above)
%
\usepackage{fancyhdr,lastpage}
\pagestyle{fancy}
%\pagestyle{empty}      % Uncomment this to get rid of page numbers
\fancyhf{}\renewcommand{\headrulewidth}{0pt}
\fancyfootoffset{\marginparsep+\marginparwidth}
\newlength{\footpageshift}
\setlength{\footpageshift}
          {0.5\textwidth+0.5\marginparsep+0.5\marginparwidth-2in}
\lfoot{\hspace{\footpageshift}%
       \parbox{4in}{\, \hfill %
                    \arabic{page} of \protect\pageref*{LastPage} % +LP
%                    \arabic{page}                               % -LP
                    \hfill \,}}

% Finally, give us PDF bookmarks
\usepackage{color,hyperref}
\definecolor{darkblue}{rgb}{0.0,0.0,0.3}
\hypersetup{colorlinks,breaklinks,
            linkcolor=darkblue,urlcolor=darkblue,
            anchorcolor=darkblue,citecolor=darkblue}

%%%%%%%%%%%%%%%%%%%%%%%% End Document Setup %%%%%%%%%%%%%%%%%%%%%%%%%%%%


%%%%%%%%%%%%%%%%%%%%%%%%%%% Helper Commands %%%%%%%%%%%%%%%%%%%%%%%%%%%%

% The title (name) with a horizontal rule under it
% (optional argument typesets an object right-justified across from name
%  as well)
%
% Usage: \makeheading{name}
%        OR
%        \makeheading[right_object]{name}
%
% Place at top of document. It should be the first thing.
% If ``right_object'' is provided in the square-braced optional
% argument, it will be right justified on the same line as ``name'' at
% the top of the CV. For example:
%
%       \makeheading[\emph{Curriculum vitae}]{Your Name}
%
% will put an emphasized ``Curriculum vitae'' at the top of the document
% as a title. Likewise, a picture could be included:
%
%   \makeheading[\includegraphics[height=1.5in]{my_picutre}]{Your Name}
%
% the picture will be flush right across from the name.
\newcommand{\makeheading}[2][]%
        {\hspace*{-\marginparsep minus \marginparwidth}%
         \begin{minipage}[t]{\textwidth+\marginparwidth+\marginparsep}%
             {\large \bfseries #2 \hfill #1}\\[-0.15\baselineskip]%
                 \rule{\columnwidth}{1pt}%
         \end{minipage}}

% The section headings
%
% Usage: \section{section name}
\renewcommand{\section}[1]{\pagebreak[3]%
    \hyphenpenalty=10000%
    \vspace{1.3\baselineskip}%
    \phantomsection\addcontentsline{toc}{section}{#1}%
    \noindent\llap{\scshape\smash{\parbox[t]{\marginparwidth}{\raggedright #1}}}%
    \vspace{-\baselineskip}\par}

% An itemize-style list with lots of space between items
\newenvironment{outerlist}[1][\enskip\textbullet]%
        {\begin{itemize}[#1,leftmargin=*]}{\end{itemize}%
         \vspace{-.6\baselineskip}}

% An environment IDENTICAL to outerlist that has better pre-list spacing
% when used as the first thing in a \section
\newenvironment{lonelist}[1][\enskip\textbullet]%
        {\begin{list}{#1}{%
        \setlength{\partopsep}{0pt}%
        \setlength{\topsep}{0pt}}}
        {\end{list}\vspace{-.6\baselineskip}}

% An itemize-style list with little space between items
\newenvironment{innerlist}[1][\enskip\textbullet]%
        {\begin{itemize}[#1,leftmargin=*,parsep=0pt,itemsep=0pt,topsep=0pt,partopsep=0pt]}
        {\end{itemize}}

% An environment IDENTICAL to innerlist that has better pre-list spacing
% when used as the first thing in a \section
\newenvironment{loneinnerlist}[1][\enskip\textbullet]%
        {\begin{itemize}[#1,leftmargin=*,parsep=0pt,itemsep=0pt,topsep=0pt,partopsep=0pt]}
        {\end{itemize}\vspace{-.6\baselineskip}}

% To add some paragraph space between lines.
% This also tells LaTeX to preferably break a page on one of these gaps
% if there is a needed pagebreak nearby.
\newcommand{\blankline}{\quad\pagebreak[3]}
\newcommand{\halfblankline}{\quad\vspace{-0.5\baselineskip}\pagebreak[3]}

% Uses hyperref to link DOI
\newcommand\doilink[1]{\href{http://dx.doi.org/#1}{#1}}
\newcommand\doi[1]{doi:\doilink{#1}}

% For \url{SOME_URL}, links SOME_URL to the url SOME_URL
\providecommand*\url[1]{\href{#1}{#1}}
% Same as above, but pretty-prints SOME_URL in teletype fixed-width font
\renewcommand*\url[1]{\href{#1}{\texttt{#1}}}

% For \email{ADDRESS}, links ADDRESS to the url mailto:ADDRESS
\providecommand*\email[1]{\href{mailto:#1}{#1}}
% Same as above, but pretty-prints ADDRESS in teletype fixed-width font
%\renewcommand*\email[1]{\href{mailto:#1}{\texttt{#1}}}

%\providecommand\BibTeX{{\rm B\kern-.05em{\sc i\kern-.025em b}\kern-.08em
%    T\kern-.1667em\lower.7ex\hbox{E}\kern-.125emX}}
%\providecommand\BibTeX{{\rm B\kern-.05em{\sc i\kern-.025em b}\kern-.08em
%    \TeX}}
\providecommand\BibTeX{{B\kern-.05em{\sc i\kern-.025em b}\kern-.08em
    \TeX}}
\providecommand\Matlab{\textsc{Matlab}}

%%%%%%%%%%%%%%%%%%%%%%%% End Helper Commands %%%%%%%%%%%%%%%%%%%%%%%%%%%

%%%%%%%%%%%%%%%%%%%%%%%%% Begin CV Document %%%%%%%%%%%%%%%%%%%%%%%%%%%%

\begin{document}
\makeheading{\href{https://aksholokhov.github.io}{Aleksei Sholokhov} \hfill \email{aksh@uw.edu}}

\section{Contact Information}

% NOTE: Mind where the & separators and \\ breaks are in the following
%       table.
%
% ALSO: \rcollength is the width of the right column of the table
%       (adjust it to your liking; default is 1.85in).
%
\newlength{\rcollength}\setlength{\rcollength}{1.85in}%
%
\begin{tabular}[t]{@{}p{\textwidth-0.2in-\rcollength}p{\rcollength}}
University of Washington & \hfill \href{https://aksholokhov.github.io}{Personal Site} \\
202 Lewis Hall  & \hfill \href{https://github.com/aksholokhov}{GitHub} \\
Seattle, 98105 & \hfill \href{https://www.linkedin.com/in/aksholokhov/}{LinkedIn} \\    
USA
\end{tabular}

\vspace{.1in}
\section{Highlights} Hi, I am Aleksei. My passion is optimization for statistical learning. I am a PhD student with strong coding skills studying Applied Math at UW. I work on feature selection, time series forecasting, and uncertainty propagation for ML models. I am also interested in deep learning, automatic differentiation tools, and HPC techniques for them. 


\vspace{.1in}
\section{Research Experience}
\href{http://www.healthdata.org}{\textbf{Institute for Health Metrics and Evaluation, UW}}
\hfill  August, 2019 - Present
\begin{outerlist}
\item[] Graduate Research Assistant in \href{https://github.com/ihmeuw-msca}{Math Science Team}. Projects: 
\begin{innerlist}
    \item \href{https://covid19.healthdata.org/united-states-of-america}{IHME Projections for COVID-19}: development and implementation of a statistical model that projects cases and deaths from COVID-19 across the world (\href{https://www.nature.com/articles/s41591-020-1132-9}{paper}).
\end{innerlist}
\end{outerlist}

\vspace{.15in}
\href{https://www.liglab.fr/en}{\textbf{Grenoble Informatics Laboratory, UGA}} \hfill March - October 2018
 \begin{outerlist}
     \item[] Visiting Research Student working on large-scale multi-label classification. Projects:
     \begin{innerlist}
     	\item \href{https://arxiv.org/abs/1811.09863}{MEMOIR}: multi-class extreme-scale SVM classifier with inexact margin \href{https://arxiv.org/abs/1811.09863}{(paper)}

     \end{innerlist}
 \end{outerlist}
 \vspace{.15in}

\href{https://cnls.lanl.gov}{\textbf{Center for Nonlinear Studies, LANL}} \hfill January, 2018
 \begin{outerlist}
     \item[] Visiting Research Student working on reinforcement learning in demand-response problems for power systems control. This work resulted in my \href{https://github.com/aksholokhov/bachelor_thesis}{bachelor thesis}. 
 \end{outerlist}
 
  \vspace{.15in}
 Computing Center of Russian Academy of Science, Moscow \hfill January, 2016 - 2018
 \begin{outerlist}
     	\item \texttt{rysearch}: an exploratory search engine and recommender system that simplifies knowledge discovery. Based on MongoDB and BigARTM.
 \end{outerlist}

 
\section{Software Development Experience}
Proficient in Python. Implemented projects in C++, MATLAB, Scala, Java.

GitHub: \href{https://github.com/aksholokhov}{aksholokhov}. Selected projects:
\begin{outerlist}
        \item \href{https://github.com/aksholokhov/skmixed}{\texttt{skmixed}}: python package for feature selection in mixed-effect models that uses a novel $\ell_0$-norm based approach. The package is fully \texttt{sklearn}-compartible. 
        \item \href{https://github.com/aksholokhov/gspack}{\texttt{gspack}}: autograder that radically simplifies creating coding assignments on Gradescope. This package is used for 5 classes at AMATH UW, for 1500+ students and counting.
\end{outerlist}
\vspace{.15in}
Notable contributions:
\begin{outerlist}
        \item \href{https://github.com/AVBelyy/Rysearch}{\texttt{rysearch}}: an exploratory search engine and recommender system. Based on MongoDB and BigARTM. 
\end{outerlist}


 
%\section{Research Experience}

%\textbf{Research Intern} \hfill {September 2017 to present}
%\begin{innerlist}
%\item[] Center for Energy Systems,\\
%        Skolkovo Institute of Science and Technology\\
%        Supervisor: Yury Maximov, Ph.D
% \end{innerlist}

\vspace{.1in}
\section{Publications}
\vspace{-.125in}
\begin{bibsection}
    \item \href{https://www.nature.com/articles/s41591-020-1132-9}{Modeling COVID-19 scenarios for the United States.} IHME Covid-19 Forecasting Team (methods contributor). \emph{Nature Medicine}, 2020.
\end{bibsection}

\vspace{.1in}
\section{Preprints}
\vspace{-.125in}
\begin{bibsection}
    \item  \href{https://arxiv.org/abs/1811.09863}{``Sparse Relaxed Regression for Covariates Selection in Mixed Models."} \textbf{Sholokhov.~A.,} Zheng.~P., Aravkin,~A.  \textit{(in progress)}.
	\item "A Scalable Data-Driven Transmission Model for COVID-19 Scenario Projections". P. Zheng, M. Bannick, \textbf{A. Sholokhov}, J. Zhang, R. Reiner, C. J.L. Murray, A. Aravkin, \textit{Currently under review in International Journal of Forecasting}, 2020.
    \item  \href{https://arxiv.org/abs/1811.09863}{``MEMOIR: Multi-class Extreme Classification with Inexact Margin."} Belyy,~A., \textbf{Sholokhov.~A.,} \textit{arXiv preprint arXiv:1811.09863 (2018)}.
\end{bibsection}
 
\vspace{.1in}
\section{Conferences}
\vspace{-.125in}
\begin{bibsection}
     \item \href{http://www.dialog-21.ru/media/4562/belyyavplusetal.pdf}{``Quality Evaluation and Improvement for Hierarchical Topic Modelling.''}, Belyy A.V., Selezniova, M.S., {\bf Sholokhov, A.,} and Vorontsov, K., \emph{ 24rd International Conference on Computational Linguistics and Intellectual Technologies} 
     \item \href{https://abitu.net/public/admin/mipt-conference/FPMI.pdf}{``Heterogeneous Aggregation of Text Data into Hierarchical Topic Models"} Selezniova, M.S., Belyy A.V., and {\bf Sholokhov, A.}, 2017. \emph{60th Scientific MIPT Conference}.
    \item \href{https://abitu.net/public/admin/mipt-conference/FPMI.pdf}{``Conditional Coordinate Descent Method for Large-Scale Statistical Estimations" 2017}. {\bf Sholokhov, A.}, \emph{60th Scientific MIPT Conference}.
\end{bibsection}

\vspace{.1in}
\section{Poster Sessions and Talks}
\vspace{-.1in}
\begin{bibsection}
    \item ``Conditional Coordinate Descent Method for Large-Scale Statistical Estimations" {\bf Sholokhov, A.}, 2017. \emph{2nd Physics Informed Machine Learning}
\end{bibsection}

\vspace{.1in}
\section{Teaching Experience}
%\textbf{Teaching Assistant} \hfill {Springs 2011--12}
Teaching Associate in University of Washington \hfill {January 2019 -- Present}
\vspace{.03in}
\begin{innerlist}
\item Calculus with Analytic Geometry II \hfill Winter 2019
\item Calculus with Analytic Geometry II \hfill Spring 2019
\item Scientific Computing in MATLAB \hfill Fall 2019 
\item Optimization: Fundamentals and Applications \hfill Winter 2020
\item High-Performance Scientific Computing \hfill Spting 2020
\end{innerlist}

\vspace{.15in}
Teaching Assistant in Remote High School of MIPT \hfill {September 2015 -- 2016}\\
\vspace{-.13in}
\begin{innerlist}
\item Mathematics, General Physics \\
\end{innerlist}

\vspace{.1in}

\section{Service and Outreach}
\href{https://www.washington.edu}{\textbf{University of Washington}},
Seattle, USA \hfill August 2018 -  June 2023
\begin{outerlist}
\item[] Diversity, Equity, and Inclusion (DEI) Committee member.
\begin{innerlist}
    \item Created 10 years Diversity Action Plan for the Department of Applied Mathematics
    \item Organized and lead on basics of diversity during DEI week.
\end{innerlist}
\end{outerlist}

\vspace{.1in}

\section{Education}

\href{https://www.washington.edu}{\textbf{University of Washington}},
Seattle, USA \hfill August 2018 -  June 2023
\begin{outerlist}
\item[] Ph.D. Student in the \href{https://amath.washington.edu}{Department of Applied Mathematics}; GPA: 3.81/4.
\begin{innerlist}
    \item Research Advisor: \href{https://uw-amo.github.io/saravkin/}{Aleksander Aravkin}
\end{innerlist}
 
\end{outerlist}
\vspace{.1in}

\href{https://mipt.ru/english/}{\textbf{Moscow Institute of Physics and Technology}},
Moscow, Russia \hfill July 2018
\begin{outerlist}

\item[] B.Sc. in Applied Mathematics and Physics, \\
        Department of Control and Applied Mathematics
        \begin{innerlist}
        \item Thesis Title: \emph{Multi-armed Bandits in Demand-Response Problems}
        \item Research Advisor:
              \href{https://cnls.lanl.gov/External/people/Yury_Maximov.php}
                   {Yury Maximov}
        \end{innerlist}
\end{outerlist}


\vspace{.1in}
\section{Awards}
Study Awards
\begin{innerlist}
\item University of Washington's Top Scholar Award, \hfill September 2018
\item MIPT Scholarship ''For Outstanding Studying Effort"  \hfill December 2015
\end{innerlist}

\halfblankline


\halfblankline

\halfblankline

\end{document}

%%%%%%%%%%%%%%%%%%%%%%%%%% End CV Document %%%%%%%%%%%%%%%%%%%%%%%%%%%%%

%----------------------------------------------------------------------%
% The following is copyright and licensing information for
% redistribution of this LaTeX source code; it also includes a liability
% statement. If this source code is not being redistributed to others,
% it may be omitted. It has no effect on the function of the above code.
%----------------------------------------------------------------------%
% Copyright (c) 2007, 2008, 2009, 2010, 2011 by Theodore P. Pavlic
%
% Unless otherwise expressly stated, this work is licensed under the
% Creative Commons Attribution-Noncommercial 3.0 United States License. To
% view a copy of this license, visit
% http://creativecommons.org/licenses/by-nc/3.0/us/ or send a letter to
% Creative Commons, 171 Second Street, Suite 300, San Francisco,
% California, 94105, USA.
%
% THE SOFTWARE IS PROVIDED "AS IS", WITHOUT WARRANTY OF ANY KIND, EXPRESS
% OR IMPLIED, INCLUDING BUT NOT LIMITED TO THE WARRANTIES OF
% MERCHANTABILITY, FITNESS FOR A PARTICULAR PURPOSE AND NONINFRINGEMENT.
% IN NO EVENT SHALL THE AUTHORS OR COPYRIGHT HOLDERS BE LIABLE FOR ANY
% CLAIM, DAMAGES OR OTHER LIABILITY, WHETHER IN AN ACTION OF CONTRACT,
% TORT OR OTHERWISE, ARISING FROM, OUT OF OR IN CONNECTION WITH THE
% SOFTWARE OR THE USE OR OTHER DEALINGS IN THE SOFTWARE.
%----------------------------------------------------------------------%
