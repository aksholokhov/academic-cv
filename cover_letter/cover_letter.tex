% Letter template which should agree largely with DIN5008A https://de.wikipedia.org/wiki/DIN_5008
% 
% Till Blaha 2023, Creative Commons
%%%%%%%%%%%%%%%%%%%%%%%%%%%%%%%%%%%%%%%%%%%%%%%%%%%%%%%%%%%%%%%%%%%%%%%%%%%%%%%%%%%%%%%%%%%%%%%%%%%%

%%% Configuration

% According to DIN5008A, the height of the right-hand location field can vary between 40 and 75mm
% you can set this value accordingly, if you need more vertical space in that field
\newcommand{\mylocheight}{40mm}

% show foldmarks?
\newcommand{\showfoldmarks}{on} % on or off

% include the preamble
\documentclass[
  fontsize=11pt,
  paper=a4,
  foldmarks=\showfoldmarks,           % Print foldmarks, all 3
  %foldmarks=Pbt,           % Print foldmarks, middle only
  %foldmarks=off,           % Print no foldmarks
  enlargefirstpage=on,    % More space on first page
  fromalign=right,        % Placement of name in letter head
  fromphone=off,           % Turn on phone number of sender
  fromemail=off,
  fromlogo=off,
  %fromrule=on,            % Rule after address in letter head
  fromrule=aftername,     % Rule after sender name in letter head
  addrfield=on,           % Adress field for envelope with window
  backaddress=on,         % Sender address in this window
  subject=afteropening,   % Placement of subject
  pagenumber=foot,
]{scrlttr2}
%\LoadLetterOption{DIN5008A} % we manually modify the geometry to comply to variant A

%%%%%%%%%%%%%%%%%%%%%%%%%%%%%%%%%%%%%%%%%%%%%%%%%%%%%%%%%%%%%%%%%%%%%%%%%%%%%%%%%%%%%%%%%%%%%%%%%%%%%%%%%%%%%



%%%%%%%%%%%%%%%%%%%%
% General Settings %
%%%%%%%%%%%%%%%%%%%%
\usepackage{blindtext} % ignore this, just for testing
\usepackage{graphicx}  % Primary graphic path
\graphicspath{{./}}
\usepackage{calc}
\usepackage{hyperref}
\usepackage{lipsum}

%%%%%%%%%%%%%%%%%%%%%%%%%%%%%%
% Font and Language settings %
%%%%%%%%%%%%%%%%%%%%%%%%%%%%%%
\usepackage[T1]{fontenc}
\usepackage{newtxtext,newtxmath}
\usepackage[utf8]{inputenc}
%\usepackage[ngerman]{babel}% http://ctan.org/pkg/babel
%\usepackage[UKenglish]{babel}% http://ctan.org/pkg/babel
\usepackage[UKenglish]{isodate} % Proper date format like 1 January 2016
\cleanlookdateon                % Proper date format like 1 January 2016

%%%%%%%%%%%%%%%%%%%%%%%%%%%%%
% General Format parameters %
%%%%%%%%%%%%%%%%%%%%%%%%%%%%%
\usepackage[lmargin=25mm, rmargin=20mm, bmargin=30mm, tmargin=25mm]{geometry}
\newcommand{\vtab}{\vspace{2.5mm}\noindent}	%initialize the vtab command
\setlength{\parskip}{3pt}					%a little space between pars
\setlength{\parindent}{0pt}					%no par indent for "modern" look

%%%%%%%%%%%%%%%
% Fonts Modes %
%%%%%%%%%%%%%%%
\setkomafont{fromname}{\sffamily \bfseries \Large \vspace{-2pt}}
\setkomafont{fromaddress}{\mdseries}	% mdseries = serif; sffamily = sans serif
\setkomafont{toname}{\mdseries}
\setkomafont{toaddress}{\mdseries}
\setkomafont{pagenumber}{\sffamily}
\setkomafont{subject}{\bfseries
mily \large}
\setkomafont{backaddress}{\mdseries}

%%%%%%%%%%%%%%%%%%%%
% Serveral "hacks" %
%%%%%%%%%%%%%%%%%%%%
%%% This is to comply to DIN5008A

%%% address field positioning
\makeatletter
  %\@setplength{backaddrheight}{0pt}% because backaddress=off
  \@setplength{toaddrhpos}{20mm}%distance from left
  \@setplength{toaddrvpos}{27mm}%distance from top
  \@setplength{toaddrheight}{45mm}%height of the addressbox
  \@setplength{toaddrwidth}{85mm}% width of the addressbox
\makeatother

%%%Din position of location field
\makeatletter
  \@setplength{locvpos}{32mm} %From top edge
  \@setplength{lochpos}{10mm} %From right edge
  \@setplength{locwidth}{85mm} % used to be 53
  \@setplength{locheight}{\mylocheight}
\makeatother

%%% fold mark positions
\makeatletter
    \@setplength{tfoldmarkvpos}{87mm}
    \@setplength{bfoldmarkvpos}{192mm}
\makeatother

%%% DIN positioning of text (assuming 5mm for date line)
\makeatletter
    \@setplength{refvpos}{35mm}
    \@addtoplength{refvpos}{\useplength{locheight}}  % reduce space before date/opening/subject/text starts
    \@setplength{subjectaftervskip}{5mm}  % reduce space below subject line
\makeatother

%%%Unindent the signature image
\makeatletter
	\@setplength{sigbeforevskip}{0pt}
\makeatother
\renewcommand*{\raggedsignature}{\raggedright}

%%% less margin at the top
\makeatletter
    \@setplength{firstheadvpos}{4mm}
\makeatother

%%%Remove Footer on first page:
%\makeatletter
%   \@setplength{firstfootvpos}{310mm}
%\makeatother


%%%%%%%%%%%%%%%%%%%%%%%%%%%%%%%%%%%%%%%%%%%%%%%%%%%%%%%%%%%%%%%%%%%%%%%%%%%%%%%%%%%%%%%%%%%%%%%%%%%%%%%%%%%%%
% Custom Header on first page
%%%%%%%%%%%%%%%%%%%%%%%%%%%%%%%%%%%%%%%%%%%%%%%%%%%%%%%%%%%%%%%%%%%%%%%%%%%%%%%%%%%%%%%%%%%%%%%%%%%%%%%%%%%%%

\newkomavar{letterheadsubtitle}
\setkomavar{firsthead}{%
  \parbox{\linewidth}{\flushright
    {\huge\sffamily\textbf{\usekomavar{fromname}}}\\
    \vspace*{-2mm}\noindent\rule{\textwidth}{0.4pt}\\
    {\large\usekomavar{letterheadsubtitle}}
  }%
}





%%%%%%%%%%%%%%%%%%%%%%%%%%%%%%%%%%%%%%%%%%%%%%%%%%%%%%%%%%%%%%%%%%%%%%%%%%%%%%%%%%%%%%%%%%%%%%%%%%%%%%%%%%%%%
% Edit only the following!                                                                                  %
%%%%%%%%%%%%%%%%%%%%%%%%%%%%%%%%%%%%%%%%%%%%%%%%%%%%%%%%%%%%%%%%%%%%%%%%%%%%%%%%%%%%%%%%%%%%%%%%%%%%%%%%%%%%%



%%% uncomment these 2 lines to make sure the boxes for address and location are not overrun
\LoadLetterOption{visualize}
%\showfields{head,address,location,refline,foot}


%%%%%%%%%%%%%%%%%%%%%%%%%% Content %%%%%%%%%%%%%%%%%%%%%%%%%%%%%%%%%%

%%%%%%%%%%%%%%%%%%%%
% Sender Variables %
%%%%%%%%%%%%%%%%%%%%
\setkomavar{fromname}{Aleksei Sholokhov}
\setkomavar{fromaddress}{202 Lewis Hall\\University of Washington\\Seattle, WA 98125\\USA}
\setkomavar{fromphone}{+1 505 557 5981}
\setkomavar{fromemail}{ak.sholokhov@gmail.com}
\setkomavar{backaddressseparator}{\ \textperiodcentered \ }
\setkomavar{signature}{\hspace{5mm}\\Aleksei Sholokhov} % with scanned signature
%\setkomavar{signature}{\vspace{10mm} \\ John Doe} % without scanned signature
\setkomavar{place}{}
\setkomavar{date}{\today}
\setkomavar{enclseparator}{: }

%%%%%%%%%%%%%%%%%%%%%%%%
% Letterhead variables %
%%%%%%%%%%%%%%%%%%%%%%%%
\setkomavar{letterheadsubtitle}{Machine Learning Researcher, PhD Student}

%%%%%%%%%%%%%%%%%%%%%%
% Location Variables %
%%%%%%%%%%%%%%%%%%%%%%

\setkomavar{location}{
\newline\vspace{-18pt}% no idea why this hack is needed to align the table to the left
\begin{tabbing}
	\hspace{35mm} \= \hspace{25mm} \= \kill % Spacing within the block
	Telephone:               \> \usekomavar{fromphone}\\
	Email:                   \> \usekomavar{fromemail}
\end{tabbing}
}


\begin{document}
%%%%%%%%%%%%%%%%%%%%%%%%%%%%%%%%%%%%
% Addressee                        %
%%%%%%%%%%%%%%%%%%%%%%%%%%%%%%%%%%%%
  \begin{letter}{%
  	Matt Patterson \\
    PARC \\
    3333 Coyote Hill Road \\
	Palo Alto, CA 94304 USA
    }
    
  %%%%%%%%%%%%%%%%%%%%%%%%%%%%%%%%%%%%%%%%%%%%%%%%%
  % Start of Actual Content of the Letter         %
  %%%%%%%%%%%%%%%%%%%%%%%%%%%%%%%%%%%%%%%%%%%%%%%%%
  
%    \setkomavar{subject}{Cover Letter for a Research Internship at Google}
    
    \opening{Dear Matt Patterson,} % 100 words
    
    
    After learning about an Internship Opportunity at PARC from Bernard Deconinck, I am reaching out to you because I believe that I am an excellent fit. Over the last five years, I gained extensive experience as a machine learning researcher in both academia and industry, and I believe that PARC is the ideal next step in my career. 

During my tenure as a research assistant at the Institute for Health Metrics and Evaluations (IHME) I developed novel feature selection methods for mixed-effect models. To create a product that perfectly fits my consumer's needs, I collaborated extensively with multiple teams on global projects like the COVID-19 Forecasting and Global Burden of Diseases Study. As a result, I published three research papers and developed one open-source package that is now widely used at the IHME and beyond. This work helped me to hone my communication skills and taught me the value of real user feedback.

In spring 2022, I developed novel methods for physics-informed reduced-order models as a machine learning researcher at Mitsubishi Electric Research Laboratories (MERL). During my internship, I drove a project from an idea to a publication in under three months and contributed significantly to another paper. I learned how to train large deep learning models and, more importantly, the importance of planning and prioritizing in time-limited research endeavors. Moreover, I discovered my passion for developing AI methods for solving hard scientific and engineering problems, which had a  profound effect on my PhD trajectory.

In summer 2022, I worked as a Machine Learning Engineer at Stripe, Inc. I was tasked to develop a model calibration pipeline for the company's transactional fraud models. It was a true challenge: I had to transform my team's vision into a project proposal and deploy the implementation to Stripe's critical production infrastructure, all in the middle of a company-wide technological transition. To succeed, I collaborated extensively with corporate leadership, fellow engineers, and the downstream consumers, giving multiple presentations a month, each tailored to different audiences. This experience taught me what it takes to deploy an ML model to critical infrastructure at scale, and this remains my proudest achievement to date. 

I am looking forward to developing cutting-edge AI technologies for some of the hardest scientific problems, while collaborating in a diverse and multi-cultural team of like-minded scientists and engineers at PARC. The letters of recommendation from the companies mentioned above are openly available on my LinkedIn profile, and examples of my coding projects are available on my GitHub page. Please don't hesitate to contact me or any of my recommenders if you have any questions. 

Have a wonderful day, and I am looking forward to hearing from you!

    
	
    \closing{Best wishes,}
    
    
    %%%%%%%%%%%%%%%%%%%%%%%
    % Enclosures          %
    %%%%%%%%%%%%%%%%%%%%%%%
    
    
    
    %%%%%%%%%%%%%%%%%%%%%%%
    % End of Tex Document %
    %%%%%%%%%%%%%%%%%%%%%%%
    
    \end{letter}
\end{document}



